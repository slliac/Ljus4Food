\subsection*{Introduction }

The Reflection\+Doc\+Block component of \mbox{\hyperlink{namespacephp_documentor}{php\+Documentor}} provides a Doc\+Block parser that is 100\% compatible with the \href{http://phpdoc.org/docs/latest}{\tt P\+H\+P\+Doc standard}.

With this component, a library can provide support for annotations via Doc\+Blocks or otherwise retrieve information that is embedded in a Doc\+Block.

\subsection*{Installation }


\begin{DoxyCode}
composer require phpdocumentor/reflection-docblock
\end{DoxyCode}


\subsection*{Usage }

In order to parse the Doc\+Block one needs a Doc\+Block\+Factory that can be instantiated using its {\ttfamily create\+Instance} factory method like this\+:


\begin{DoxyCode}
$factory  = \mbox{\hyperlink{classphp_documentor_1_1_reflection_1_1_doc_block_factory_a220aa312016b6fa0c9a6c4bf61a7ead4}{\(\backslash\)phpDocumentor\(\backslash\)Reflection\(\backslash\)DocBlockFactory::createInstance}}
      ();
\end{DoxyCode}


Then we can use the {\ttfamily create} method of the factory to interpret the Doc\+Block. Please note that it is also possible to provide a class that has the {\ttfamily get\+Doc\+Comment()} method, such as an object of type {\ttfamily Reflection\+Class}, the create method will read that if it exists.


\begin{DoxyCode}
$docComment = <<<DOCCOMMENT
DOCCOMMENT;

$docblock = $factory->create($docComment);
\end{DoxyCode}


The {\ttfamily create} method will yield an object of type {\ttfamily \textbackslash{}\mbox{\hyperlink{namespacephp_documentor}{php\+Documentor}}\textbackslash{}Reflection\textbackslash{}Doc\+Block} whose methods can be queried\+:


\begin{DoxyCode}
\textcolor{comment}{// Contains the summary for this DocBlock}
$summary = $docblock->getSummary();

\textcolor{comment}{// Contains \(\backslash\)phpDocumentor\(\backslash\)Reflection\(\backslash\)DocBlock\(\backslash\)Description object}
$description = $docblock->getDescription();

\textcolor{comment}{// You can either cast it to string}
$description = (string) $docblock->getDescription();

\textcolor{comment}{// Or use the render method to get a string representation of the Description.}
$description = $docblock->getDescription()->render();
\end{DoxyCode}


\begin{quote}
For more examples it would be best to review the scripts in the \href{/examples}{\tt {\ttfamily /examples} folder}.\end{quote}
